\section{Fazit}
Abschließend lässt sich festhalten, dass unsere Gruppe trotz anfänglicher Schwierigkeiten zu einem verlässlichen Endprodukt gekommen ist. Lino und ich haben Vieles gemeinsam besprochen und Ideen entwickelt, jedoch von Anfang an eine gute Arbeitsteilung gehabt, indem wir die theoretischen Vorüberlegungen noch gemeinsam angestellt haben, ab da sich aber jeder zunehmend auf einen Themenbereich fokussiert hat. Während ich hauptsächlich für unsere GUI und die Implementierung neuer, für die Simulation nützlicher Funktionen zuständig war, hat sich Lino viel um die Implementierung und Optimierung des \glqq{}Kern-Algorithmus\grqq{} gekümmert. \\
Durch das Praktikum konnte ich meine Kenntnisse in JavaScript deutlich erweitern und auch das erste Mal an einem etwas größeren Projekt mit mehreren verschiedenen Arbeitsaufteilungen/Gruppen arbeiten. Das hat mir gezeigt, wie wichtig eine gute Abstimmung und Kommunikation in der Praxis ist. \\
Insgesamt haben wir natürlich - im Nachhinein betrachtet, vermeidbare - Fehler gemacht, aber ich denke, das ist normal und hat uns auch viel gelehrt. Mit unserer Simulation bin ich aber durchaus zufrieden.