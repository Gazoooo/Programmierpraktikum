\section{Einleitung}
\label{src:einleitung}
In diesem Bericht geht es um die Inhaltlichen Schwerpunkte der \glqq{}Simulations-Gruppe\grqq{} im Rahmen des Programmierpraktikums Computergrafik im WiSe24/25 der Universität Osnabrück. \\
Dabei bestand diese Gruppe aus zwei Mitgliedern: Mir, Gustav Otzen, und Lino Viets. Manches haben wir zusammen erarbeitet, uns für unsere Berichte aber unterschiedliche Themenschwerpunkte herausgesucht, über die wir hauptsächlich schreiben; An manchen Stellen sind Doppelungen aber trotzdem nicht auszuschließen. \\
Mein Bericht setzt sich daher aus den folgenden drei Hauptkapiteln zusammen, wobei ich persönlich besonders an dem dritten und großen Teilen des zweiten Punktes gearbeitet habe:

\begin{itemize}
\item Kommunikation mit den anderen Gruppen
\item Entwurf der graphischen Benutzeroberfläche
\item Entwurf einer flüssigen Animation
\end{itemize}

Nachfolgend werden diese Kapitel der Reihe nach näher erläutert. Davor wird zwecks Kontextualisierung das Ziel des Praktikums im folgenden Absatz einmal kurz umrissen.

\subsection{Zielgebung}
Das Programmierpraktikum beschäftigte sich mit der Entwicklung einer Zeichenmaschine, welche durch kontinuierliche Kurven ein beliebiges Bild in schwarz-weiß auf einer Unterlage (Whiteboard, Papier) malt. Dazu wurden wir in verschiedene Gruppen eingeteilt, die alle zusammen gearbeitet haben, um am Ende dieses Ziel  zu erreichen. Jede Gruppe ist dafür verschiedene Kernpunkte und Probleme angegangen, wie beispielsweise das 3D-Drucken der benötigten Ressourcen oder das Entwerfen eines Algorithmus, der ein gegebenes Bild in Bezierkurven aufteilt. Wie schon oben erwähnt, bestand meine Aufgabe darin, eine Simulation für diesen Zeichenprozess zu  entwickeln. Lino und ich haben uns dazu für die Programmiersprache JavaScript (mit HTML und CSS) entschieden; Näheres zu den Gründen steht in Linos Bericht.
